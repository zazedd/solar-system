\chapter*{Agradecimentos}
\label{chap:ack}
\vspace{0.7cm}

Após terminar o projeto, é impossível não refletir sobre a imensa contribuição que recebemos ao longo do tempo. Neste capítulo, gostaríamos de expressar os nossos mais profundos 
agradecimentos ao professor Abel João Padrão Gomes e a todos os membros do grupo que tornaram esta experiência educacional significativa. Ao professor, expressamos a nossa mais sincera gratidão. O seu comprometimento apaixonado com o ensino e a sua dedicação incansável ao compartilhamento de conhecimento foram fundamentais para o nosso aprendizado. Além disso, não podemos deixar de expressar nossa gratidão a cada membro do grupo. Cada um de nós trouxe uma perspectiva única, habilidades distintas e um compromisso inabalável com a realização dos objetivos do trabalho.
As nossas reuniões foram marcadas por colaboração, criatividade e respeito mútuo, elementos essenciais para o nosso sucesso coletivo. A troca de ideias, o trabalho árduo e a camaradagem que cultivámos juntos foram fundamentais para enfrentar os desafios que encontrámos ao longo deste percurso académico. Cada membro do grupo desempenhou um papel vital, contribuindo não apenas para a conclusão do trabalho, mas também para o crescimento individual de cada um. É impossível deixar de agradecer também aos nossos familiares, que se fizeram presentes, nos dias bons e maus, e em momento algum nos deixaram cair, apoiando-nos sempre para nos mantermos no bom sentido, nesta que é a nossa caminhada para o sucesso profissional.

\chapter*{Resumo}
\label{chap:res}
\vspace{0.7cm}

Pensar que cada ser humano é apenas uma peça pequena de um grande puzzle que constitui a nossa galáxia deixa muitas pessoas fascinadas, pela nossa inferioridade comparando com a complexidade e o tamanho do que nos rodeia. Para essas pessoas é ainda mais fascinante ver os elementos do sistema solar com os seus próprios olhos, com ajuda de telescópios e imagens partilhadas por satélites, apesar de nem sempre ser fácil de conseguir ver o que queremos. Mas e se conseguissemos representar o nosso sistema solar num computador? Seria muito mais fácil e barato ver tudo aquilo que queremos, sem saírmos das nossas próprias casas, para além de nos fornecer uma maior imersão.
Ao longo deste documento vamos explicar detalhadamente como foi desenvolvida e implementada a nossa versão do sistema solar, com os vários planetas e informações sobre os mesmos. Vamos explicar também a magia que fez todas as físicas dos planetas funcionar, como utilizámos bibliotecas de \ac{UI} para criar menus que possibilitam alterar partes do sistema em tempo real, e muito mais.

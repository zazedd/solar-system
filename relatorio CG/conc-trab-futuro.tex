\chapter{Conclusões e Trabalho Futuro}
\label{chap:conc-trab-futuro}

\section{Conclusões Principais}
\label{sec:conc-princ}

Dando por terminado o desenvolvimento deste projeto, concluímos que conseguimos alcançar todos os nossos objetivos, obtendo assim uma simulação realista do nosso sistema solar. Refletindo sobre tudo o que foi feito ao longo destes meses de trabalho, percebemos o quão importante é a área da \ac{CG} e o quão importante foi este projeto para a nosso aprendizagem, já que com ele conseguimos aprender muito para além do que foi possível nas aulas.
Concluímos também que fizémos um excelente trabalho, com uma boa organização e divisão do trabalho entre todos os elementos do grupo, estando muito orgulhosos do resultado final.

\section{Trabalho Futuro}
\label{sec:trab-futuro}

Neste projeto, devido a problemas de falta de tempo livre para investirmos mais no projeto, acabámos por não ter tempo para realizar a renderização de texto com auxílio de \textit{bitmaps} ou \textit{billboards}, mas ficando assim como um objetivo futuro para aprimoramento do projeto. Para suprir a falta disto, criámos labels por via de menus com ajuda da biblioteca \ac{IMGUI}.

\noindent
Para além disso, seria bastante interessante adicionar à nossa simulação outros planetas e galáxias distantes, ou até mesmo simular buracos negros e "brincar" com a gravidade e todas as leis da física que conhecemos hoje.
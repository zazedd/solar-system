\chapter{Introdução}
\label{chap:intro}

\section{Âmbito, Enquadramento e Motivação}
\label{sec:amb} 
% CADA SECÇÃO DEVE TER UM LABEL
% CADA FIGURA DEVE TER UM LABEL
% CADA TABELA DEVE TER UM LABEL

% Esta parte da Introdução, tem como intuito responder a questões como: 
% O que é este documento? 
% De uma forma sintética, qual o contexto deste projeto (convergindo depois para a resposta às duas questões seguintes)? 
% Qual a área em que se insere este projeto?
% Quais as duas sub-áreas em que se insere este projeto? 
% Por que é importante abordar o problema abordado neste projeto? 
% A resposta deve ser impessoal. Não se trata da motivação pessoal, mas sim da motivação técnica a alimentar o projeto. A resposta a cada uma das questões apresentadas deve ser dada em um parágrafo. Note que o Enquadramento e a Motivação podem, por vezes, ser secções diferentes.

Neste documento vamos descrever como foi feito, organizado e dividido o nosso projeto da \ac{UC} de \ac{CG}, onde escolhemos desenvolver um Sistema Solar, o projeto que achámos ser mais interessante de fazer pela sua dificuldade e pelas possibilidades de conceitos que poderíamos implementar para o complementar.

\noindent
O desenvolvimento deste projeto tornou-se importante para consolidar os básicos de \ac{CG} e para aprender outros conceitos muito importantes e que nos deixaram fascinados pela sua complexidade e pelos resultados obtidos quando estes se aplicam. 

\section{Objetivos do Trabalho}
\label{sec:obj}

Com este projeto, o nosso maior objetivo foi melhorar e afinar
os nossos conhecimentos sobre \ac{CG}, pelo que a sua realização foi fundamental, assim como toda a investigação realizada para o mesmo.

\noindent
Também tivemos como objetivos a aprendizagem de conceitos que desconhecíamos e que se mostraram ser muito interessantes e importantes, tanto nesta área como noutras, ganhando assim um conhecimento extra que certamente será útil no nosso futuro profissional.

\section{Organização do Documento}
\label{sec:organ}

De modo a descrever fielmente o trabalho que foi feito, este documento encontra-se estruturado da seguinte forma:

\begin{enumerate}
\item O primeiro capítulo -- \textbf{Introdução} -- onde falámos sobre o projeto de um modo geral, referindo os seus objetivos e a sua organização.
\item O segundo capítulo -- \textbf{Desenvolvimento e Implementação} -- onde se falou sobre como se desenvolveu a aplicação e o que foi usado para desenvolver a mesma.
\item O terceiro capítulo -- \textbf{Conclusões e Trabalho Futuro} -- onde se falou sobre as aprendizagens que ficaram com o desenvolvimento deste projeto e o que pode ser feito no futuro para o melhorar.
\end{enumerate}